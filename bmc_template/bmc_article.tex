%% BioMed_Central_Tex_Template_v1.06
%%                                      %
%  bmc_article.tex            ver: 1.06 %
%                                       %

%%IMPORTANT: do not delete the first line of this template
%%It must be present to enable the BMC Submission system to
%%recognise this template!!

%%%%%%%%%%%%%%%%%%%%%%%%%%%%%%%%%%%%%%%%%
%%                                     %%
%%  LaTeX template for BioMed Central  %%
%%     journal article submissions     %%
%%                                     %%
%%          <8 June 2012>              %%
%%                                     %%
%%                                     %%
%%%%%%%%%%%%%%%%%%%%%%%%%%%%%%%%%%%%%%%%%


%%%%%%%%%%%%%%%%%%%%%%%%%%%%%%%%%%%%%%%%%%%%%%%%%%%%%%%%%%%%%%%%%%%%%
%%                                                                 %%
%% For instructions on how to fill out this Tex template           %%
%% document please refer to Readme.html and the instructions for   %%
%% authors page on the biomed central website                      %%
%% http://www.biomedcentral.com/info/authors/                      %%
%%                                                                 %%
%% Please do not use \input{...} to include other tex files.       %%
%% Submit your LaTeX manuscript as one .tex document.              %%
%%                                                                 %%
%% All additional figures and files should be attached             %%
%% separately and not embedded in the \TeX\ document itself.       %%
%%                                                                 %%
%% BioMed Central currently use the MikTex distribution of         %%
%% TeX for Windows) of TeX and LaTeX.  This is available from      %%
%% http://www.miktex.org                                           %%
%%                                                                 %%
%%%%%%%%%%%%%%%%%%%%%%%%%%%%%%%%%%%%%%%%%%%%%%%%%%%%%%%%%%%%%%%%%%%%%

%%% additional documentclass options:
%  [doublespacing]
%  [linenumbers]   - put the line numbers on margins

%%% loading packages, author definitions

%\documentclass[twocolumn]{bmcart}% uncomment this for twocolumn layout and comment line below
\documentclass{bmcart}

%%% Load packages
%\usepackage{amsthm,amsmath}
%\RequirePackage{natbib}
%\RequirePackage[authoryear]{natbib}% uncomment this for author-year bibliography
\RequirePackage{hyperref}
\usepackage[utf8]{inputenc} %unicode support
%\usepackage[applemac]{inputenc} %applemac support if unicode package fails
%\usepackage[latin1]{inputenc} %UNIX support if unicode package fails
\usepackage{footnote}

%%%%%%%%%%%%%%%%%%%%%%%%%%%%%%%%%%%%%%%%%%%%%%%%%
%%                                             %%
%%  If you wish to display your graphics for   %%
%%  your own use using includegraphic or       %%
%%  includegraphics, then comment out the      %%
%%  following two lines of code.               %%
%%  NB: These line *must* be included when     %%
%%  submitting to BMC.                         %%
%%  All figure files must be submitted as      %%
%%  separate graphics through the BMC          %%
%%  submission process, not included in the    %%
%%  submitted article.                         %%
%%                                             %%
%%%%%%%%%%%%%%%%%%%%%%%%%%%%%%%%%%%%%%%%%%%%%%%%%


\def\includegraphic{}
\def\includegraphics{}



%%% Put your definitions there:
\startlocaldefs
\endlocaldefs


%%% Begin ...
\begin{document}

%%% Start of article front matter
\begin{frontmatter}

\begin{fmbox}
\dochead{Research}

%%%%%%%%%%%%%%%%%%%%%%%%%%%%%%%%%%%%%%%%%%%%%%
%%                                          %%
%% Enter the title of your article here     %%
%%                                          %%
%%%%%%%%%%%%%%%%%%%%%%%%%%%%%%%%%%%%%%%%%%%%%%

\title{Clinical Application of a Targeted Sequencing Panel for Germline Pharmacogenomic Testing in Formalin-fixed Paraffin-embedded Tumours}

%%%%%%%%%%%%%%%%%%%%%%%%%%%%%%%%%%%%%%%%%%%%%%
%%                                          %%
%% Enter the authors here                   %%
%%                                          %%
%% Specify information, if available,       %%
%% in the form:                             %%
%%   <key>={<id1>,<id2>}                    %%
%%   <key>=                                 %%
%% Comment or delete the keys which are     %%
%% not used. Repeat \author command as much %%
%% as required.                             %%
%%                                          %%
%%%%%%%%%%%%%%%%%%%%%%%%%%%%%%%%%%%%%%%%%%%%%%

\author[
   addressref={aff1},                   % id's of addresses, e.g. {aff1,aff2}
   % noteref={n1},                        % id's of article notes, if any
   email={eyap@bcgsc.ca}   % email address
]{\inits{SQ}\fnm{Shyong Quin} \snm{Yap}}
\author[
   addressref={aff1,aff2},
	 corref={aff1},                       % id of corresponding address, if any
   email={akarsan@bcgsc.ca}
]{\inits{A}\fnm{Aly} \snm{Karsan}}

%%%%%%%%%%%%%%%%%%%%%%%%%%%%%%%%%%%%%%%%%%%%%%
%%                                          %%
%% Enter the authors' addresses here        %%
%%                                          %%
%% Repeat \address commands as much as      %%
%% required.                                %%
%%                                          %%
%%%%%%%%%%%%%%%%%%%%%%%%%%%%%%%%%%%%%%%%%%%%%%

\address[id=aff1]{%                           % unique id
  \orgname{British Columbia Cancer Research Centre}, % university, etc
  \street{675 West 10th Ave},                     %
  \postcode{V5Z 1L3}                              % post or zip code
  \city{Vancouver, BC},                           % city
  \cny{Canada}                                    % country
}
\address[id=aff2]{%
  \orgname{Department of Pathology and Laboratory Medicine, University of British Columbia},
  \street{Random Street},
  \postcode{Random Post Code}
  \city{Vancouver, BC},
  \cny{Canada}
}

%%%%%%%%%%%%%%%%%%%%%%%%%%%%%%%%%%%%%%%%%%%%%%
%%                                          %%
%% Enter short notes here                   %%
%%                                          %%
%% Short notes will be after addresses      %%
%% on first page.                           %%
%%                                          %%
%%%%%%%%%%%%%%%%%%%%%%%%%%%%%%%%%%%%%%%%%%%%%%

\begin{artnotes}
%\note{Sample of title note}     % note to the article
%\note[id=n1]{Equal contributor} % note, connected to author
\end{artnotes}

\end{fmbox}% comment this for two column layout

%%%%%%%%%%%%%%%%%%%%%%%%%%%%%%%%%%%%%%%%%%%%%%
%%                                          %%
%% The Abstract begins here                 %%
%%                                          %%
%% Please refer to the Instructions for     %%
%% authors on http://www.biomedcentral.com  %%
%% and include the section headings         %%
%% accordingly for your article type.       %%
%%                                          %%
%%%%%%%%%%%%%%%%%%%%%%%%%%%%%%%%%%%%%%%%%%%%%%

\begin{abstractbox}

\begin{abstract} % abstract
\parttitle{Background} %if any
Because the tumour genome contains germline information, clinical laboratories can leverage tumour genomic testing to perform initial screening for clinically relevant germline variants such as variants in pharmacogenomic (PGx) genes. Subsequently, only patients with potential germline PGx variants would be subjected to downstream germline testing. While this procedure is more cost-effective than sequencing and analyzing normal DNA for every patient, the concordance of germline PGx variants between tumour and matched normal specimens must be evaluated to determine the feasibility of detecting germline PGx variants in tumour DNA.

\parttitle{Results} %if any

\parttitle{Conclusions} %if any


\end{abstract}

%%%%%%%%%%%%%%%%%%%%%%%%%%%%%%%%%%%%%%%%%%%%%%
%%                                          %%
%% The keywords begin here                  %%
%%                                          %%
%% Put each keyword in separate \kwd{}.     %%
%%                                          %%
%%%%%%%%%%%%%%%%%%%%%%%%%%%%%%%%%%%%%%%%%%%%%%

\begin{keyword}
\kwd{Tumour sequencing}
\kwd{Targeted next-generation sequencing panel}
\kwd{Germline pharmacogenomics testing}
\kwd{Formalin-fixed paraffin-embedded tumours}
\end{keyword}

% MSC classifications codes, if any
%\begin{keyword}[class=AMS]
%\kwd[Primary ]{}
%\kwd{}
%\kwd[; secondary ]{}
%\end{keyword}

\end{abstractbox}
%
%\end{fmbox}% uncomment this for twcolumn layout

\end{frontmatter}

%%%%%%%%%%%%%%%%%%%%%%%%%%%%%%%%%%%%%%%%%%%%%%
%%                                          %%
%% The Main Body begins here                %%
%%                                          %%
%% Please refer to the instructions for     %%
%% authors on:                              %%
%% http://www.biomedcentral.com/info/authors%%
%% and include the section headings         %%
%% accordingly for your article type.       %%
%%                                          %%
%% See the Results and Discussion section   %%
%% for details on how to create sub-sections%%
%%                                          %%
%% use \cite{...} to cite references        %%
%%  \cite{koon} and                         %%
%%  \cite{oreg,khar,zvai,xjon,schn,pond}    %%
%%  \nocite{smith,marg,hunn,advi,koha,mouse}%%
%%                                          %%
%%%%%%%%%%%%%%%%%%%%%%%%%%%%%%%%%%%%%%%%%%%%%%

%%%%%%%%%%%%%%%%%%%%%%%%% start of article main body
% <put your article body there>

%%%%%%%%%%%%%%%%
%% Background %%
%%
\section*{Background}

Tumour profiling using next-generation sequencing (NGS) technologies has been increasingly integrated into oncologic care to detect targetable somatic mutations and personalize treatments for cancer patients. Although analysis of tumour-normal paired samples is required to accurately discriminate between somatic and germline variants, most clinical laboratories only sequence tumour samples to minimize cost and turnaround time \cite{Raymond2016}. However, genomic analyses of tumours can also reveal secondary genomic findings, which are germline information that may have clinical implications for patients and their family members \cite{Raymond2016}. In fact, several studies demonstrated that a germline cancer-predisposing variant is present in 3-10\% of patients undergoing tumour-normal sequencing \cite{Raymond2016,Meric-Bernstam2016,Schrader2015,Jones2015}. Therefore, clinical laboratories providing tumour genomic testing must be equipped to perform germline confirmatory testing on potential germline variants or be prepared to refer such cases to external services.

Because the tumour genome contains germline information, clinical laboratories can leverage tumour genomic testing to perform initial screening for clinically relevant germline variants such as variants in pharmacogenomic (PGx) genes. Subsequently, a similar framework for validating secondary germline findings can be applied, in which only patients with potential germline PGx variants are subjected to downstream germline testing. This procedure for germline PGx testing is more cost-effective because it does not require processing, sequencing, and analysis of normal DNA for every patient. The ability to implement germline PGx testing at a reduced cost can significantly benefit patient care because these variants cause functional changes in drug targets and drug disposition proteins (proteins involved drug metabolism and transport), thereby contributing to inter-patient differences in chemotherapeutic response \cite{McLeod2013}. Hence, such genomic information can be used to guide the selection of chemotherapeutic drugs and optimization of drug dosage for cancer patients, leading to improved safety and efficacy of treatment and reduced risk of toxicity \cite{McLeod2013}.

Detection of genomic alterations in tumour DNA is also faced with technical challenges conferred by formalin-fixed paraffin-embedded (FFPE) tumour specimens \cite{Do2015,Wong2014}. Tumour biopsies are often formalin-fixed to preserve tissue morphology for histological examination and to enable storage at room temperature; however, formalin fixation causes DNA fragmentation and base modifications, which pose difficulties in using DNA extracted from FFPE tumours for clinical genomic testing \cite{Do2015,Wong2014}. Fragmentation damage caused by formalin fixation leads to reduced template DNA for PCR amplification, thereby affecting the efficiency of amplicon-based NGS testing \cite{Do2015,Wong2014}. Furthermore, the degree of DNA fragmentation was shown to be higher in tissues from older FFPE blocks and tissues fixed with formalin of lower pH \cite{Do2015}. Formalin fixation is also problematic because it gives rise to depurination, which generates abasic sites, and cytosine deamination resulting in C$>$T/G$>$A transitions \cite{Do2015}. These forms of formalin-induced DNA damage contributes to the presence of sequence artifacts in FFPE specimens, which can be inaccurately identified as real genomic alterations.

In this study, we assessed the concordance of germline PGx variants between tumour and matched normal DNA by analyzing amplicon-based targeted NGS data from 213 patients with tumour-normal paired samples. While matched normal DNA was derived from peripheral blood, tumour DNA was extracted from FFPE tumour blocks; thus, we compared the quality metrics of sequencing data between tumour and blood specimens and evaluated the prevalence of formalin-induced DNA damages to address the impact of formalin fixation on amplicon-based NGS testing. We demonstrated that germline PGx variants can be identified with high sensitivity and precision in FFPE tumour DNA using a clinical targeted sequencing panel.

%\cite{koon,oreg,khar,zvai,xjon,schn,pond,smith,marg,hunn,advi,koha,mouse}

\section*{Methods}

\subsection*{Patient Samples}
Blood and FFPE tumour specimens were acquired from 213 patients who provided informed consent for The OncoPanel Pilot (TOP) study, a pilot study to optimize the OncoPanel, which is an amplicon-based targeted NGS panel for solid tumours, and assess its application for guiding disease management and therapeutic intervention. Patients in the TOP study are those with advanced cancers including colorectal cancer, lung cancer, melanoma, gastrointestinal stromal tumor (GIST), and other cancers (\autoref{cancertypes}). The age of paraffin block for tumour specimens ranges from 18 to 5356 days with a median of 274 days.

\subsection*{OncoPanel (Solid Tumour NGS Panel)}
The OncoPanel is offered by the British Columbia Cancer Agency (BCCA) for clinical genomic testing of coding exons and clinically relevant hotpots of 20 cancer-related genes and six PGx genes: \textit{DPYD}, \textit{GSTP1}, \textit{MTHFR}, \textit{TYMP}, \textit{TYMS}, and \textit{UGT1A1}. Full list of genes and gene reference models for the OncoPanel is presented in \autoref{genemodel}. Primers were designed by RainDance Technologies (Lexington, MA) using the GRCh37/hg19 reference sequence to generate 416 amplicons between 100 bp and 250 bp in size, which interrogate $\sim$ 20 kb of target bases. Target regions of the six PGx genes in the OncoPanel were assayed by 49/416 amplicons. Complete lists of primers and amplicons are provided in the Supplemental Materials.

\subsection*{Sample Preparation, Library Construction, and Illumina Sequencing}
Genomic DNA was extracted from blood and FFPE tumour specimens using the Gentra Autopure LS DNA preparation platform and QIAamp DNA FFPE tissue kit (Qiagen, Hilden, Germany) respectively. The extracted DNA was sheared according to a previously described protocol \cite{Bosdet2013} to attain approximate sizes of 3 kb followed by PCR primer merging, amplification of target regions, and adapter ligation using the Thunderstorm NGS Targeted Enrichment System (RainDance Technologies, Lexington, MA) as per manufacturer's protocol. Barcoded amplicons were sequenced with the Illumina MiSeq system for paired end sequencing with a v2 250-bp kit (Illumina, San Diego, CA).

\subsection*{Variant Calling Pipeline}
Reads that passed the Illumina Chastity filter were aligned to the GRCh37/hg19 human reference genome using the BWA mem algorithm (version 0.5.9) with default parameters, and the alignments were processed and converted to the BAM format using SAMtools (version 0.1.18). Variant calling was performed with the SAMtools \texttt{mpileup} function \texttt{(samtools mpileup -BA -d 500000 -L 500000 -q 1)} to generate pileup files for all target bases followed by the VarScan2 \texttt{mpileup2cns} (version 2.3.6) function with parameter thresholds of variant allele frequency $\geq$ 10\% and Phred base quality $\geq$ 20 \texttt{(--min-var-freq 0.1 --p-value 0.01 --strand-filter 0 --output-vcf --variants --min-avg-qual 20)}. Variant calls were filtered using the VarScan2 \texttt{fpfilter} function with fraction of variant reads from each strand $\geq$ 0.1 and default thresholds for other parameters. SnpEff (version 4.2) was used for variant annotation and effect prediction whereas the SnpSift package in SnpEff was used to annotate variants with databases such as dbSNP (b138), COSMIC (version 70), 1000 Genomes Project, ClinVar, and ExAC (release 0.3) for interpretation.

\subsection*{Data Analysis}
Specifics on how each analysis was performed.


\section*{Results}

\subsection*{Formalin fixation affects amplicon generation and sequencing}

\subsection*{Coverage depth and quality are highly variable in FFPE tumours}

\subsection*{Formalin-induced sequence artifacts are prevalent at low variant allele frequency}

\subsection*{Germline PGx variants can be detected with high sensitivity and precision in FFPE tumours}

\subsection*{Discordant germline PGx variants are caused by low coverage sites}


\section*{Discussion}

\section*{Conclusions}

%%%%%%%%%%%%%%%%%%%%%%%%%%%%%%%%%%%%%%%%%%%%%%
%%                                          %%
%% Backmatter begins here                   %%
%%                                          %%
%%%%%%%%%%%%%%%%%%%%%%%%%%%%%%%%%%%%%%%%%%%%%%

\begin{backmatter}

\section*{Competing interests}
  The authors declare that they have no competing interests.

\section*{Author's contributions}
    Text for this section \ldots

\section*{Acknowledgements}
  Text for this section \ldots
%%%%%%%%%%%%%%%%%%%%%%%%%%%%%%%%%%%%%%%%%%%%%%%%%%%%%%%%%%%%%
%%                  The Bibliography                       %%
%%                                                         %%
%%  Bmc_mathpys.bst  will be used to                       %%
%%  create a .BBL file for submission.                     %%
%%  After submission of the .TEX file,                     %%
%%  you will be prompted to submit your .BBL file.         %%
%%                                                         %%
%%                                                         %%
%%  Note that the displayed Bibliography will not          %%
%%  necessarily be rendered by Latex exactly as specified  %%
%%  in the online Instructions for Authors.                %%
%%                                                         %%
%%%%%%%%%%%%%%%%%%%%%%%%%%%%%%%%%%%%%%%%%%%%%%%%%%%%%%%%%%%%%

% if your bibliography is in bibtex format, use those commands:
\bibliographystyle{bmc-mathphys} % Style BST file (bmc-mathphys, vancouver, spbasic).
\bibliography{/Users/evayap/Documents/biblitex_mendeley/pgx_ms.bib}      % Bibliography file (usually '*.bib' )
% for author-year bibliography (bmc-mathphys or spbasic)
% a) write to bib file (bmc-mathphys only)
% @settings{label, options="nameyear"}
% b) uncomment next line
%\nocite{label}

% or include bibliography directly:
% \begin{thebibliography}
% \bibitem{b1}
% \end{thebibliography}

%%%%%%%%%%%%%%%%%%%%%%%%%%%%%%%%%%%
%%                               %%
%% Figures                       %%
%%                               %%
%% NB: this is for captions and  %%
%% Titles. All graphics must be  %%
%% submitted separately and NOT  %%
%% included in the Tex document  %%
%%                               %%
%%%%%%%%%%%%%%%%%%%%%%%%%%%%%%%%%%%

%%
%% Do not use \listoffigures as most will included as separate files

\section*{Figures}
  \begin{figure}[h!]
  \caption{\csentence{Workflow for Library Construction and Sequencing.}
      A short description of the figure content
      should go here.}
      \end{figure}

\begin{figure}[h!]
  \caption{\csentence{Sample figure title.}
      Figure legend text.}
      \end{figure}

%%%%%%%%%%%%%%%%%%%%%%%%%%%%%%%%%%%
%%                               %%
%% Tables                        %%
%%                               %%
%%%%%%%%%%%%%%%%%%%%%%%%%%%%%%%%%%%

%% Use of \listoftables is discouraged.
%%
\section*{Tables}

\begin{table}[!h]
\caption{Distribution of cancer types in the TOP cohort.}\label{cancertypes}
      \begin{tabular}{lccc}
        \hline
        Cancer Type & Number of Cases & Percentage (\%) \\ \hline
        Colorectal & 97 & 46 \\
        Lung & 59 & 28 \\
        Melanoma & 18 & 8 \\
				Other* & 17 & 8 \\
				GIST & 7 & 3 \\
				Sarcoma & 4 & 2 \\
				Neuroendocrine & 4 & 2 \\
				Cervical & 2 & 0.9 \\
				Ovarian & 2 & 0.9 \\
				Breast & 2 & 0.9 \\
				Unknown & 1 & 0.5 \\ \hline
      \end{tabular} \\
{\small *This category includes thyroid, peritoneum, sarcomatoid carcinoma of lung, Fallopian tube, gastric, endometrial, anal, salivary gland, pancreas, and small bowel cancers.}
\end{table}

\normalsize
\begin{table}
    \caption{Gene Reference Models for Genes in the OncoPanel.}\label{genemodel}
    \centering
    \begin{tabular}{ l l l }
		\hline
    Gene & Protein & Reference Model \\
		\hline
    AKT1 & Protein kinase B & NM\_001014431.1 \\
    ALK & Anaplastic lymphoma receptor tyrosine kinase & NM\_004304.3 \\
    BRAF & Serine/threonine-protein kinase B-Raf & NM\_004333.4 \\
    DPYD & Dihydropyrimidine dehydrogenase & NM\_000110.3 \\
    EGFR & Epidermal growth factor receptor & NM\_005228.3 \\
    ERBB2 & Receptor tyrosine-protein kinase erbB-2 & NM\_001005862.1 \\
    GSTP1 & Glutathione S-rransferase pi 1 & NM\_000852.3 \\
    HRAS & GTPase HRas & NM\_005343.2 \\
    IDH1 & Isocitrate dehydrogenase 1 & NM\_005896.2 \\
    IDH2 & Isocitrate dehydrogenase 2 & NM\_002168.2 \\
    KIT & Tyrosine-protein kinase Kit & NM\_000222.2 \\
    KRAS & KRas proto-oncogene GTPase & NM\_033360.2 \\
    MAPK1 & Mitogen-activated protein kinase 1 & NM\_002745.4 \\
    MAP2K1 & Mitogen-activated protein kinase kinase 1 & NM\_002755.3 \\
    MTHFR & Methylenetetrahydrofolate reductase & NM\_005957.4 \\
    MTOR & Serine/threonine-protein kinase mTOR & NM\_004958.3 \\
    NRAS & Neuroblastoma RAS viral oncogene homolog & NM\_002524.3 \\
    PDGFRA & Platelet-derived growth factor receptor alpha & NM\_006206.4 \\
    PIK3CA & Phosphatidylinositol-4,5-bisphosphate 3-kinase catalytic subunit alpha & NM\_006218.2 \\
    PTEN & Phosphatase and tensin homolog & NM\_000314.4 \\
    STAT1 & Signal transducer and activator of transcription 1 & NM\_007315.3 \\
    STAT3 & Signal transducer and activator of transcription 3 & NM\_139276.2 \\
    TP53 & Tumor protein P53 & NM\_000546.5 \\
    TYMP & Thymidine phosphorylase & NM\_001113755.2 \\
    TYMS & Thymidylate synthetase & NM\_001071.2 \\
    UGT1A1 & Uridine diphosphate (UDP)-glucuronosyl transferase 1A1 & NM\_000463.2\\
    \hline
    \end{tabular}
\end{table}

%%%%%%%%%%%%%%%%%%%%%%%%%%%%%%%%%%%
%%                               %%
%% Additional Files              %%
%%                               %%
%%%%%%%%%%%%%%%%%%%%%%%%%%%%%%%%%%%

\section*{Additional Files}
  \subsection*{Additional file 1 --- Sample additional file title}
    Additional file descriptions text (including details of how to
    view the file, if it is in a non-standard format or the file extension).  This might
    refer to a multi-page table or a figure.

  \subsection*{Additional file 2 --- Sample additional file title}
    Additional file descriptions text.


\end{backmatter}
\end{document}
\endinput

\subsection*{Study Design}
This study involved retrospective analysis of tumour-normal sequencing data from 213 patients diagnosed with various advanced cancers. The matched normal specimens for this study were derived from peripheral blood, which served as controls for assessment of formalin-induced DNA damages in FFPE tumours and gold standard specimens for detecting germline variants.

\subsubsection*{DNA Input and Amplicon Yield}

\paragraph*{Sub-sub-sub heading for section}
Text for this sub-sub-sub-heading \ldots
