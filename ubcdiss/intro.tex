%% The following is a directive for TeXShop to indicate the main file
%%!TEX root = diss.tex

\chapter{Introduction}
\label{ch:Introduction}

%%%%%%%%%%%%%%%%%%%%%%%%%%%%%%%%%%%%%%%%%%%%%%%%%%%%%%%%%%%%%%%%%%%%%%
\section{The Evolution of Molecular Diagnostics in Cancer}
\label{sec:The Evolution of Molecular Diagnostics in Cancer}

Cancer is a 


%%%%%%%%%%%%%%%%%%%%%%%%%%%%%%%%%%%%%%%%%%%%%%%%%%%%%%%%%%%%%%%%%%%%%%
\section{Next-generation Sequencing Technologies}
\label{sec:Next-generationSequencingTechnologies}


%%%%%%%%%%%%%%%%%%%%%%%%%%%%%%%%%%%%%%%%%%%%%%%%%%%%%%%%%%%%%%%%%%%%%%
\section{Applications of Next-generation Sequencing}
\label{sec:ApplicationsofNext-generationSequencing}

\subsection{Targeted Sequencing}
Capture-based, amplicon-based etc.

\subsection{Whole Exome Sequencing}

\subsection{Whole Genome Sequencing}

%%%%%%%%%%%%%%%%%%%%%%%%%%%%%%%%%%%%%%%%%%%%%%%%%%%%%%%%%%%%%%%%%%%%%%
\section{Bioinformatics Tools for Variant Calling}
\label{sec:BioinformaticsToolsforVariantCalling}

\subsection{Types of Genomic Alterations}
There are different types of genomic alterations.

\subsection{Variant Calling Pipeline}

\subsection{Variant Calling Algorithms}

\subsection{Variant Curation and Interpretation}

%%%%%%%%%%%%%%%%%%%%%%%%%%%%%%%%%%%%%%%%%%%%%%%%%%%%%%%%%%%%%%%%%%%%%%
\section{Germline Variant Calling in The Tumour Genome}
\label{sec:GermlineVariantCallinginTheTumourGenome}

\subsection{Incidental Findings}
The application of next-generation sequencing (NGS) technologies for tumour profiling has been increasingly integrated into oncologic care to detect targetable somatic mutations and personalize treatments for cancer patients. Although analysis of tumour-normal paired samples is required to accurately discriminate between somatic and germline variants, most clinical laboratories only sequence tumour samples to minimize cost and turnaround time \cite{Raymond2016}. However, genomic analyses of tumours can also reveal secondary genomic findings, which are germline information that may have clinical implications for patients and their family members \cite{Raymond2016}. In fact, several studies demonstrated that a germline cancer-predisposing variant is present in 3-10\% of patients undergoing tumour-normal sequencing \cite{Raymond2016,Meric-Bernstam2016,Schrader2015,Jones2015}. Therefore, clinical laboratories providing tumour genomic testing must be equipped to perform germline confirmatory testing on potential germline variants or be prepared to refer such cases to external services.

\subsection{Pharmacogenomic Variants}
MMQS higher means more mismatches in the supporting reads
Because the tumour genome contains germline information, clinical laboratories can leverage tumour genomic testing to perform initial screening for clinically relevant germline variants such as variants in pharmacogenomic (PGx) genes. Subsequently, a similar framework for validating secondary germline findings can be applied, in which only patients with potential germline PGx variants are subjected to downstream germline testing. This procedure for germline PGx testing is more cost-effective because it does not require processing, sequencing, and analysis of normal DNA for every patient. The ability to implement germline PGx testing at a reduced cost can significantly benefit patient care because these variants cause functional changes in drug targets and drug disposition proteins (proteins involved in drug metabolism and transport), thereby contributing to inter-patient differences in chemotherapeutic response \cite{McLeod2013}. Hence, such genomic information can be used to guide the selection of chemotherapeutic drugs and optimization of drug dosage for cancer patients, leading to improved safety and efficacy of treatment and reduced risk of toxicity \cite{McLeod2013}.

\subsection{Challenges}
Detection of genomic alterations in tumour DNA is also faced with technical challenges conferred by formalin-fixed paraffin-embedded (FFPE) tumour specimens \cite{Do2015,Wong2014}. Tumour biopsies are often formalin-fixed to preserve tissue morphology for histological examination and to enable storage at room temperature; however, formalin fixation causes DNA fragmentation and base modifications, which pose difficulties in using DNA extracted from FFPE tumours for clinical genomic testing \cite{Do2015,Wong2014}. Fragmentation damage caused by formalin fixation leads to reduced template DNA for PCR amplification, thereby affecting the efficiency of amplicon-based NGS testing \cite{Do2015,Wong2014}. Furthermore, the degree of DNA fragmentation was shown to be higher in tissues from older FFPE blocks and tissues fixed with formalin of lower pH \cite{Do2015}. Formalin fixation is also problematic because it gives rise to depurination, which generates abasic sites, and cytosine deamination resulting in C$>$T/G$>$A transitions \cite{Do2015}. These forms of formalin-induced DNA damage contribute to the presence of sequence artifacts in FFPE specimens, which can be inaccurately identified as real genomic alterations.


%%%%%%%%%%%%%%%%%%%%%%%%%%%%%%%%%%%%%%%%%%%%%%%%%%%%%%%%%%%%%%%%%%%%%%
\section{Objectives}
\label{sec:Objectives}
