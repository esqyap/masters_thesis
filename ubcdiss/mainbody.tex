%% The following is a directive for TeXShop to indicate the main file
%%!TEX root = diss.tex

\chapter{Germline Pharmacogenomics Testing in Formalin-Fixed Paraffin-Embedded Tumours}
\label{ch:GermlinePharmacogenomicsTestinginFormalin-FixedParaffin-EmbeddedTumours}

%%%%%%%%%%%%%%%%%%%%%%%%%%%%%%%%%%%%%%%%%%%%%%%%%%%%%%%%%%%%%%%%%%%%%%
\section{Overview}
\label{sec:Overview}

Application of genome information to guide patient management and therapeutic intervention holds great potential in improving oncology care. One of the driving forces that led to clinical feasibility of genomic sequencing is the advent of massively parallel sequencing (MPS) technologies, which enabled sensitive and accurate sequencing of more target genes with less DNA in a cost-effective and timely manner. At present, various MPS approaches are entering, or have entered the clinic such as targeted sequencing panels, whole exome sequencing, and whole genome sequencing, which create the opportunity to further develop novel clinical biomarkers in addition to screening for biomarkers with established clinical utility.

In the context of cancer,
Clinical biomarkers can be classified as diagnostic, prognostic, predictive, and pharmacogenomic (PGx). In the context of cancer, both somatic Germline variants that affect influence drug response are PGx biomarkers.

In the context of cancer, PGx biomarkers are

Clinical molecular laboratories are rapidly adopting and leveraging the advances in massively parallel sequencing (MPS) technologies for germline and tumour profiling. have driven the clinical use of genomic information to guide patient management and therapeutic intervention in oncology care. The ability of MPS to sensitively and accurately sequence more target genes with less DNA in a cost-effective and timely manner perfectly meet the clinical reality which is to do more with less.

reduced cost, time and laborhigh throughput nature, decreased sequencing cost, and increased sensitivity, and ability to Genomics-driven cancer medicine aims is driven by the advances in massively parallel sequencing (MPS) technologies, the reduced cost of genome sequencing, and development of bioinformatics analytic tools. This emerging framework in the oncology care aims to use genomics information to inform

The Oncopanel is a clinical targeted sequencing panel for solid tumours provided by the CCG at the BCCA. In addition to somatic mutations, it screens for germline variants in PGx genes such as DPYD, GSTP1, MTHFR, TYMP, TYMS, and UGT1A1 (Table 1). Detection of germline PGx variants is essential for chemotherapy selection and optimization of treatment dosage and duration. The Oncopanel is also delivered as a single sample clinical assay in which genetic variants are detected in DNA from FFPE tumours. However, formalin fixation causes DNA fragmentation and base transition artifacts (i.e. C$>$T and G$>$A). Hence, I investigated whether germline PGx variants could be detected with high sensitivity and precision in FFPE tumour DNA compared to blood DNA which is the gold standard for germline variant calling.

%%%%%%%%%%%%%%%%%%%%%%%%%%%%%%%%%%%%%%%%%%%%%%%%%%%%%%%%%%%%%%%%%%%%%%
\section{Methods}
\label{sec:Methods}

%%%%%%%%%%%%%%%%%%%%%%%%%%%%%%%%%%%%%%%%%%%%%%%%%%%%%%%%%%%%%%%%%%%%%%
\section{Results and Discussion}
\label{sec:Results and Discussion}
