%% The following is a directive for TeXShop to indicate the main file
%%!TEX root = diss.tex

\chapter{Variant Calling in Formalin-fixed Paraffin-embedded Tumours}
\label{ch:VariantCallinginFormalin-fixedParaffin-embeddedTumours}

%%%%%%%%%%%%%%%%%%%%%%%%%%%%%%%%%%%%%%%%%%%%%%%%%%%%%%%%%%%%%%%%%%%%%%
\section{Overview}
\label{sec:Overview}

The Oncopanel is a clinical targeted sequencing panel for solid tumours provided by the CCG at the BCCA. In addition to somatic mutations, it screens for germline variants in PGx genes such as DPYD, GSTP1, MTHFR, TYMP, TYMS, and UGT1A1 (Table 1). Detection of germline PGx variants is essential for chemotherapy selection and optimization of treatment dosage and duration. The Oncopanel is also delivered as a single sample clinical assay in which genetic variants are detected in DNA from FFPE tumours. However, formalin fixation causes DNA fragmentation and base transition artifacts (i.e. C$>$T and G$>$A). Hence, I investigated whether germline PGx variants could be detected with high sensitivity and precision in FFPE tumour DNA compared to blood DNA which is the gold standard for germline variant calling.
