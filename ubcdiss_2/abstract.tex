%% The following is a directive for TeXShop to indicate the main file
%%!TEX root = diss.tex

\chapter{Abstract}

Germline alterations have clinical implications for cancer patients and their families. Because the tumour genome contains both germline and somatic variants, clinical tumour sequencing presents an opportunity for pre-screening of germline variants. This framework is time- and cost-effective because only patients with potential germline variants are referred to downstream confirmatory testing. A key challenge in this framework is distinguishing between germline and somatic variants in the tumour. Tumour specimens are also commonly formalin-fixed paraffin-embedded (FFPE), which induces DNA damage that interferes with molecular testing. To determine the feasibility of leveraging tumour sequencing for identifying germline variants, the usability of FFPE DNA for germline testing must be evaluated and an approach to differentiate between germline and somatic variants must be established. We retrospectively analyzed clinical amplicon sequencing data from 213 patients with tumour and matched normal samples. We assessed formalin-induced DNA damage by comparing efficiency in amplicon enrichment and sequencing results of FFPE DNA to DNA isolated from blood, a gold standard for germline testing. Although predominant forms of formalin-induced DNA damage like fragmentation and cytosine deamination were detectable, we determined that the discrepancies were minor and could be minimized through methods such as using shorter amplicons and avoiding long-term storage of FFPE blocks. We also found that 98.0\% of germline alterations identified in the blood were retained in the tumours, indicating that tumour DNA is a reliable source for germline variant calling. Finally, we applied variant allele frequency (VAF) thresholds to delineate germline and somatic variants in tumour-only analyses. We reported that VAF cut-offs of 15\%, 20\%, 25\%, and 30\% would correctly identify 99\%, 98\%, 96\%, and 94\% of germline alterations, respectively. However, the same VAF cut-offs would also erroneously submit 14\%, 12\%, 11\%, and 10\% of somatic mutations (false positives), respectively, for follow-up germline testing. This underscores the high sensitivity and precision of using VAF to discriminate between germline and somatic variants. Collectively, our results demonstrate that leveraging tumour sequencing for identifying germline variants could be a practical, cost-efficient approach for providing germline testing.


\endinput
We reported that a 30\% VAF cut-off would correctly identify 94\% of germline alterations but erroneously include 10\% of somatic mutations (false positives) for follow-up germline testing.


Genomic analyses of tumours can not only reveal actionable somatic mutations, but also germline variants with clinical implications for patients and their families. Thus, clinical tumour sequencing presents an opportunity to perform initial screening for germline variants. This framework is cost-saving because only patients with potential germline variants would be referred to downstream confirmatory testing. A key challenge in implementing this framework is distinguishing between germline and somatic variants in the tumour genome. Furthermore, tumour specimens are typically formalin-fixed paraffin-embedded (FFPE), which induces DNA damage that poses challenges in molecular testing. To determine the feasibility of leveraging clinical tumour sequencing for germline testing, the usability of FFPE DNA for germline testing and an approach to differentiate between germline and somatic variants in tumour-only analyses must be evaluated.

We retrospectively analyzed clinical amplicon sequencing data from 213 patients with tumour and matched normal samples. To evaluate the usability of FFPE DNA for germline testing, we characterized formalin-induced DNA damage by comparing efficiency in amplicon enrichment and sequencing results of FFPE DNA to blood, a gold standard for germline testing. Although formalin-induced DNA damage including fragmentation and cytosine deamination were detectable, we determined that these discrepancies were either technically negligible or could be minimized using available methods. We also found that 93.8\% of germline alterations identified in blood were present with the same allelic status in FFPE tumours. This implied that the majority of germline genetic changes were retained in the tumours, demonstrating the reliability of using tumour DNA for germline variant calling. Finally, we assessed the application of variant allele fraction (VAF) thresholds to delineate germline and somatic variants in tumour-only analyses. We reported that a VAF cut-off of 30\% would correctly identify 94\% of germline alterations, while erroneously submit 10\% of false positives, which were somatic mutations, for follow-up germline testing. This underscores the high sensitivity and precision of using VAF to discriminate between germline and somatic variants. Our results collectively demonstrate an invaluable finding in clinical genomics wherein leveraging FFPE tumour sequencing for identification of germline variants could be a practical, cost-efficient approach for providing germline testing.

Tumour sequencing has been rapidly adopted into oncologic care to inform disease management and treatment with targeted therapies.

While sequencing of tumour-normal pairs would enable simultaneous identification of clinically important germline variants, matched normal samples are often not obtained in clinical practice.

A framework that leverages clinical tumour sequencing for germline testing is cost-saving because only patients with potential germline variants would be referred to downstream confirmatory testing.

However, this would require evaluating usability of FFPE DNA for germline testing and establishing an approach to distinguish between germline and somatic variants in tumour-only analyses.
