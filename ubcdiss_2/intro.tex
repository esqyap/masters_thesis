%% The following is a directive for TeXShop to indicate the main file
%%!TEX root = diss.tex

\chapter{Introduction}
\label{ch:Introduction}

%%%%%%%%%%%%%%%%%%%%%%%%%%%%%%%%%%%%%%%%%%%%%%%%%%%%%%%%%%%%%%%%%%%%%%
\section{Cancer as a Genetic Disease}
\label{sec:CancerasaGeneticDisease}

Cancers are diseases defined by unrestrained proliferation of cells that are capable of invading normal tissues and metastasizing to other parts of the body. Early studies between the late nineteenth and early twentieth centuries by David von Hansemann and Theodor Boveri suggested that genetic alterations may contribute to oncogenesis. In von Hansemann's analysis of tumour samples, he observed features of aberrant cell division, which he speculated to be contributing factors of unequal distribution of chromosomes in the tumour cells. Boveri explored the connection between defective cell division and tumour formation by inducing abnormal chromosome segregation in sea urchin eggs and observing the outcome of these cells. While most cases of chromosomal imbalance resulted in cell death, Boveri reported that there were cases in which cell survival was followed by uncontrolled cell growth. These findings led Boveri to surmise that the improper combination of genetic materials could sustain the proliferative ability of tumour cells. In particular, tumour cells were likely to retain chromatin parts with growth stimulatory effects or remove those with growth inhibitory effects. Boveri's speculations pertaining genetic materials that function as stimulators or inhibitors of cell growth were consistent with the modern understanding of oncogenes and tumour-suppressor genes, respectively.

Oncogenes and tumour-suppressor genes are two categories of cancer-causing genes that play a central role in cancer initiation and progression. Prototype oncogenes (proto-oncogenes), the normal counterparts of oncogenes, encode proteins that promote cell growth and survival, as well as inhibit cell differentiation. When proto-oncogenes sustain dominant gain-of-function mutations, they become oncogenes, giving rise to constitutively active or overexpressed protein products that induce malignant transformation of the cells. The first oncogene, v-\textit{src}, was identified by Peyton Rous in a retrovirus that causes sarcoma in chickens, and the virus was later named the Rous sarcoma virus after its discoverer. Subsequently, the proto-oncogene c-\textit{src}, a homologue of v-\textit{src}, was discovered, but unlike its mutant form, the protein encoded by c-\textit{src} was not constitutively active. The discovery that proto-oncogenes exist in healthy cells led to the recognition that normal cellular genes are capable of gaining oncogenic potential through acquiring gene mutations. This breakthrough in cancer research consequently catalyzed the identification of more proto-oncogenes, providing an enhanced understanding of cellular signalling pathways.

Tumour-suppressor genes, the other important type of cancer-causing genes, can be separated into gatekeepers and caretakers. Gatekeepers are involved in regulating cell-cycle checkpoints and mutations in gatekeepers would directly result in cancer development. On the other hand, caretakers are DNA repair proteins, and mutant caretakers could indirectly cause malignant transformation of cells by inducing accumulation of mutations, thereby increasing the probability that mutations would occur in oncogenes and tumour-suppressor genes. Unlike oncogenes, mutations in tumour-suppressor genes are recessive-acting, meaning that two mutant alleles are required for the tumour-suppressor gene to become oncogenic. Alfred Knudson was the first to propose that mutant tumor-suppressor genes function in a recessive fashion, a notable concept later known as Knudson's two-hit hypothesis.  Knudson's statistical model demonstrated that familial retinoblastoma, a pediatric eye cancer, was consistent with a one-hit curve, meaning that a single mutation was sufficient to cause tumour formation. Conversely, non-familial retinoblastoma was consistent with a two-hit curve, meaning that two mutations were involved in the disease. These findings implied that disease carriers, who inherited one mutant allele, only require the loss of the remaining functional allele to drive tumour formation, whereas non-carriers require two mutation events to trigger the development of tumours.

Some bullshit about stepwise progression...

%%%%%%%%%%%%%%%%%%%%%%%%%%%%%%%%%%%%%%%%%%%%%%%%%%%%%%%%%%%%%%%%%%%%%%
\section{The Evolution of Molecular Diagnostics in Cancer}
\label{sec:The Evolution of Molecular Diagnostics in Cancer}

Although early studies have implicated the causative role of genetic alterations in cancer development, initial classification of cancers were based on the primary site of the tumour. This long-standing classification was caused by limitations in technologies and tools, as well as by the clinical classification required by surgical management, which was the initial mainstay of clinical oncology. Subsequently, microscopy-based classification of disease further delineated cancer subsets based on histologic differences. For example, aggressiveness or risk of relapse was retrospectively linked to histologic grading as a prognostic biomarker, such as Gleason and Bloom-Richardson for prostate and breast cancer, respectively. The histologic classification was advanced with assessment of prototypic surface markers (immunohistochemistry), gross markers for lymphoid subsets have heavily influenced the diagnostic classification of lymphoma. Characterization of chromosomal abnormalities in leukemia and sarcoma has aided in the diagnosis of disease subsets as well as prognostic assessment.

In conclusion, the diagnosis of cancer has undergone a paradigm shift. No longer is cancer diagnosed only based on morphological parameters. More and more the diagnostic algorithm is supported by immunohistochemical and molecular alterations at the DNA, mRNAs, miRNAs and proteomic level. Multiple platforms and high throughput technological advances enable faster and cheaper analysis of all these as well as the whole genome. This is having a significant impact on how medicine is now being practiced in a personalized approach leading to the development of precision medicine based on pharmacogenomics. It is being realized that a tumor may not be characterized by a single gene alteration but by a panel of ‘signature’ genomic alterations leading to targeted therapeutic strategies and surveillance based on the tumor specific alterations. The ultimate goal of cancer diagnosis in personalized medicine would be to identify the correct diagnosis and guide the therapy so that every patient received precision medicine that is the right drug at the right dose.

Cancer is a group of diseases characterized by uncontrolled proliferation of cells that are capable of normal tissue invasion and metastasis to distant organs.

Oncogenes encode for proteins that stimulate cell proliferation and survival as well as inhibit cell differentiation leading to oncogenesis. The first human oncogene was identified based on homology with

Mutations in oncogenes are typically dominant, which means . The first human oncogenes was discovered Conversely, tumour-suppressor genes encode for proteins that inhibit cell proliferation and survival as well as stimulate cell differentiation. Mutations in oncogenes are typically dominant whereas mutations in tumour-suppressor genes are recessive.

In the normal cell, proto-oncogenes stimulate proliferation and inhibit differentiation and apoptosis while the opposite is true for tumour suppressors. Proto-oncogenes are usually dominant, meaning that only one gain-of-function mutation is required to activate the oncogene, thereby causing cancer. Conversely, tumour suppressor genes are usually recessive and two loss-of-function events are required.

Boveri and von Hansemann (1890-1914) - oncogenes and tumour suppressors
Philadelphia translocation (1960)
Two hit hypothesis, retinoblastoma (1971) - germline and somatic mutations
HRAS point mutation (1982)
Eric Fearon and Bert Vogelstein find specific sequential mutations in carcinoma (1990) - multi-step process, caretakers and gatekeepers
Types of mutations/gene changes - SNVs, indels, SVs
Driver vs. passenger mutations - evolutionary process, selective growth advantage, CSCs
Frequency and pathway-based: three main pathways

The pathogenesis of cancer is caused by genetic abnormalities
Although fundamentally known to arise from genetic mutations, the disease paradigm has expanded to include aberrant epigenetic mechanisms as a contributing factor to oncogenesis.
The understanding of cancer pathogenesis has expanded been increasing over the years and a disorder that was fundamentally known to arise from genetic mutations this group of disorders which have been fundamentally known Cancer has been fundamentally known as a genetic disease defined by abnormal proliferation of cells.
Our understanding of cancer pathogenesis has been expanding  Although the understanding of cancer pathogenesis has been expanding, Cancer has been fundamentally known as a genetic disease.

%%%%%%%%%%%%%%%%%%%%%%%%%%%%%%%%%%%%%%%%%%%%%%%%%%%%%%%%%%%%%%%%%%%%%%
\section{The Era of Precision Oncology}
\label{sec:TheEraofPrecisionOncology}

Cancer diagnosis and treatment have been revolutionized by advances in next-generation sequencing and bioinformatics tools, which contributed to an enhanced understanding of the mutational landscapes of various cancers. In the era of precision oncology, tumours can be sequenced, somatic mutations can be detected, and patients can be treated with targeted therapies or referred to on-going clinical trials. However, the tumour genome also consists of germline information that may have clinical implications for patients and their families.


%%%%%%%%%%%%%%%%%%%%%%%%%%%%%%%%%%%%%%%%%%%%%%%%%%%%%%%%%%%%%%%%%%%%%%
\section{Next-generation Sequencing Technologies}
\label{sec:Next-generationSequencingTechnologies}

Next-generation sequencing


%%%%%%%%%%%%%%%%%%%%%%%%%%%%%%%%%%%%%%%%%%%%%%%%%%%%%%%%%%%%%%%%%%%%%%
\section{Applications of Next-generation Sequencing}
\label{sec:ApplicationsofNext-generationSequencing}

\subsection{Targeted Sequencing}
Capture-based, amplicon-based etc.

\subsection{Whole Exome Sequencing}

\subsection{Whole Genome Sequencing}

%%%%%%%%%%%%%%%%%%%%%%%%%%%%%%%%%%%%%%%%%%%%%%%%%%%%%%%%%%%%%%%%%%%%%%
\section{Variant Calling Pipeline}
\label{sec:VariantCallingPipeline}


%%%%%%%%%%%%%%%%%%%%%%%%%%%%%%%%%%%%%%%%%%%%%%%%%%%%%%%%%%%%%%%%%%%%%%
\section{Germline Variants in The Tumour Genome}
\label{sec:GermlineVariantCallinginTheTumourGenome}

\subsection{Incidental Findings}
The application of next-generation sequencing (NGS) technologies for tumour profiling has been increasingly integrated into oncologic care to detect targetable somatic mutations and personalize treatments for cancer patients. Although analysis of tumour-normal paired samples is required to accurately discriminate between somatic and germline variants, most clinical laboratories only sequence tumour samples to minimize cost and turnaround time \cite{Raymond2016}. However, genomic analyses of tumours can also reveal secondary genomic findings, which are germline information that may have clinical implications for patients and their family members \cite{Raymond2016}. In fact, several studies demonstrated that a germline cancer-predisposing variant is present in 3-10\% of patients undergoing tumour-normal sequencing \cite{Raymond2016,Meric-Bernstam2016,Schrader2015,Jones2015}. Therefore, clinical laboratories providing tumour genomic testing must be equipped to perform germline confirmatory testing on potential germline variants or be prepared to refer such cases to external services.

\subsection{Pharmacogenomic Variants}
MMQS higher means more mismatches in the supporting reads
Because the tumour genome contains germline information, clinical laboratories can leverage tumour genomic testing to perform initial screening for clinically relevant germline variants such as variants in pharmacogenomic (PGx) genes. Subsequently, a similar framework for validating secondary germline findings can be applied, in which only patients with potential germline PGx variants are subjected to downstream germline testing. This procedure for germline PGx testing is more cost-effective because it does not require processing, sequencing, and analysis of normal DNA for every patient. The ability to implement germline PGx testing at a reduced cost can significantly benefit patient care because these variants cause functional changes in drug targets and drug disposition proteins (proteins involved in drug metabolism and transport), thereby contributing to inter-patient differences in chemotherapeutic response \cite{McLeod2013}. Hence, such genomic information can be used to guide the selection of chemotherapeutic drugs and optimization of drug dosage for cancer patients, leading to improved safety and efficacy of treatment and reduced risk of toxicity \cite{McLeod2013}.

\subsection{Challenges}
Detection of genomic alterations in tumour DNA is also faced with technical challenges conferred by formalin-fixed paraffin-embedded (FFPE) tumour specimens \cite{Do2015,Wong2014}. Tumour biopsies are often formalin-fixed to preserve tissue morphology for histological examination and to enable storage at room temperature; however, formalin fixation causes DNA fragmentation and base modifications, which pose difficulties in using DNA extracted from FFPE tumours for clinical genomic testing \cite{Do2015,Wong2014}. Fragmentation damage caused by formalin fixation leads to reduced template DNA for PCR amplification, thereby affecting the efficiency of amplicon-based NGS testing \cite{Do2015,Wong2014}. Furthermore, the degree of DNA fragmentation was shown to be higher in tissues from older FFPE blocks and tissues fixed with formalin of lower pH \cite{Do2015}. Formalin fixation is also problematic because it gives rise to depurination, which generates abasic sites, and cytosine deamination resulting in C$>$T/G$>$A transitions \cite{Do2015}. These forms of formalin-induced DNA damage contribute to the presence of sequence artifacts in FFPE specimens, which can be inaccurately identified as real genomic alterations.

%%%%%%%%%%%%%%%%%%%%%%%%%%%%%%%%%%%%%%%%%%%%%%%%%%%%%%%%%%%%%%%%%%%%%%
\section{ACCE Model Process for Evaluating Genetic Tests}
\label{sec:ACCEModelProcessforEvaluatingGeneticTests}



%%%%%%%%%%%%%%%%%%%%%%%%%%%%%%%%%%%%%%%%%%%%%%%%%%%%%%%%%%%%%%%%%%%%%
%%%%%%%%%%%%%%%%%%%%%%%%%%%%%%%%%%%%%%%%%%%%%%%%%%%%%%%%%%%%%%%%%%%%%

\begin{figure}[H]
	\centering
	\includegraphics[scale=0.6]{acce_wheel.png}
	\caption{}
	\label{fig:acce_wheel}
\end{figure}

%%%%%%%%%%%%%%%%%%%%%%%%%%%%%%%%%%%%%%%%%%%%%%%%%%%%%%%%%%%%%%%%%%%%%
%%%%%%%%%%%%%%%%%%%%%%%%%%%%%%%%%%%%%%%%%%%%%%%%%%%%%%%%%%%%%%%%%%%%%


%%%%%%%%%%%%%%%%%%%%%%%%%%%%%%%%%%%%%%%%%%%%%%%%%%%%%%%%%%%%%%%%%%%%%%
\section{Objectives}
\label{sec:Objectives}

This thesis aims to determine whether potential germline alterations can be accurately identified in FFPE tumours without the use of matched normal samples for follow-up testing. We performed analytic validation of a clinical amplicon-based targeted sequencing panel for FFPE solid tumours by comparison with sequencing of blood DNA, which is the gold standard for germline testing. Our objectives include (1) assessing the degree of formalin-induced DNA damage in FFPE DNA, (2) determining the concordance of germline alterations between blood and FFPE tumours, and (3) evaluating the use of VAF thresholds to distinguish germline alterations from somatic mutations in tumour-only analyses, as well as establishing a VAF cut-off that would maximize true positive rate of identifying germline alterations in FFPE tumours and minimize referral of somatic mutations (false positives) to downstream germline testing.
