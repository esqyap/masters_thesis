%% The following is a directive for TeXShop to indicate the main file
%%!TEX root = diss.tex

\chapter{Lay Summary}

Hereditary genetic changes have clinical impacts on cancer patients and their families. Tumours contain tumour-specific and inherited genetic variations. Using tumour DNA for pre-screening of hereditary variants is cost-saving because only patients with potential hereditary variants would require follow-up. Follow-up testing involves analyzing blood or saliva to confirm the presence of the potential hereditary variants before making clinical decisions. A key challenge in implementing this approach is differentiating between tumour-specific and inherited variants in the tumour DNA. Furthermore, the commonly-used tumour fixative, formalin, induces DNA damage, which interferes with using tumour DNA for genetic testing. We showed that the effects of formalin on tumour DNA were minor, and we established a highly sensitive and precise method for separating hereditary variants from tumour-specific mutations. Our findings implied that extracting hereditary information from tumour DNA analysis could serve as a practical, cost-effective approach to providing hereditary genetic testing in the clinic.
