%% The following is a directive for TeXShop to indicate the main file
%%!TEX root = diss.tex

\chapter{Materials and Methods}
\label{ch:Materialsandmethods}

%%%%%%%%%%%%%%%%%%%%%%%%%%%%%%%%%%%%%%%%%%%%%%%%%%%%%%%%%%%%%%%%%%%%%%
\section{Patient Samples}
\label{sec:PatientSamples}

Blood and FFPE tumour specimens were acquired from 213 patients who provided informed consent for The OncoPanel Pilot (TOP) study, a pilot study to optimize the OncoPanel, which is an amplicon-based targeted NGS panel for solid tumours. The TOP study also aims to assess the OncoPanel's application for guiding disease management and therapeutic intervention. Patients in the TOP study are those with advanced cancers including colorectal cancer, lung cancer, melanoma, gastrointestinal stromal tumour (GIST), and other cancers (\autoref{tbl:cancertypes}). The age of paraffin block for tumour specimens ranges from 18 to 5356 days with a median of 274 days.

\begin{table}[H]
\caption{Distribution of cancer types in the TOP cohort.}
\label{tbl:cancertypes}
\centering
      \begin{tabular}{lccc}
        \hline
        Cancer Type & Number of Cases & Percentage (\%) \\ \hline
        Colorectal & 97 & 46 \\
        Lung & 59 & 28 \\
        Melanoma & 18 & 8 \\
				Other* & 17 & 8 \\
				GIST & 7 & 3 \\
				Sarcoma & 4 & 2 \\
				Neuroendocrine & 4 & 2 \\
				Cervical & 2 & 0.9 \\
				Ovarian & 2 & 0.9 \\
				Breast & 2 & 0.9 \\
				Unknown & 1 & 0.5 \\ \hline
      \end{tabular} \\
			\vspace{0.5cm}
\justify
{\small *This category includes thyroid, peritoneum, lung sarcomatoid carcinoma, Fallopian tube, gastric, endometrial, squamous cell carcinoma, anal, salivary gland, peritoneal epithelial mesothelioma, adenoid cystic carcinoma, pancreas, breast, gall bladder, parotid epithelial myoepithelial carcinoma, and small bowel cancers.}
\end{table}

%%%%%%%%%%%%%%%%%%%%%%%%%%%%%%%%%%%%%%%%%%%%%%%%%%%%%%%%%%%%%%%%%%%%%%
\section{Sample Preparation, Library Construction, and Illumina Sequencing}
\label{sec:SamplePreparation,LibraryConstruction,andIlluminaSequencing}

Genomic DNA was extracted from blood and FFPE tumour specimens using the Gentra Autopure LS DNA preparation platform and QIAamp DNA FFPE tissue kit (Qiagen, Hilden, Germany), respectively. The extracted DNA was sheared according to a previously described protocol \cite{Bosdet2013} to attain approximate sizes of 3 kb followed by PCR primer merging, amplification of target regions, and adapter ligation using the Thunderstorm NGS Targeted Enrichment System (RainDance Technologies, Lexington, MA) as per manufacturer's protocol. Barcoded amplicons were sequenced with the Illumina MiSeq system for paired end sequencing with a v2 250-bp kit (Illumina, San Diego, CA).


%%%%%%%%%%%%%%%%%%%%%%%%%%%%%%%%%%%%%%%%%%%%%%%%%%%%%%%%%%%%%%%%%%%%%
\section{OncoPanel (Targeted NGS Panel for Solid Tumours)}
\label{sec:OncoPanel}

The OncoPanel assesses coding exons and clinically relevant hotpots of 15 cancer predisposing genes and six PGx genes that can predict chemotherapeutic response. Primers were designed by RainDance Technologies (Lexington, MA) using the GRCh37/hg19 reference sequence to generate 416 amplicons between 56 bp and 288 bp in size, which interrogate $\sim$ 20 kb of target bases. Complete list of genes and gene reference models for the OncoPanel is presented in \autoref{tbl:genemodel}, whereas OncoPanel target regions and amplicons are presented in \autoref{tbl:amplicons_target_regions}.

\begin{table}[H]
    \caption{Gene Reference Models for Genes in the OncoPanel.}\label{genemodel}
    \centering
    \begin{tabular}{ l l l }
		\hline
    Gene & Protein & Reference Model \\
		\hline
    AKT1 & Protein kinase B & NM\_001014431.1 \\
    ALK & Anaplastic lymphoma receptor tyrosine kinase & NM\_004304.3 \\
    BRAF & Serine/threonine-protein kinase B-Raf & NM\_004333.4 \\
    DPYD & Dihydropyrimidine dehydrogenase & NM\_000110.3 \\
    EGFR & Epidermal growth factor receptor & NM\_005228.3 \\
    ERBB2 & Receptor tyrosine-protein kinase erbB-2 & NM\_001005862.1 \\
    GSTP1 & Glutathione S-rransferase pi 1 & NM\_000852.3 \\
    HRAS & GTPase HRas & NM\_005343.2 \\
    IDH1 & Isocitrate dehydrogenase 1 & NM\_005896.2 \\
    IDH2 & Isocitrate dehydrogenase 2 & NM\_002168.2 \\
    KIT & Tyrosine-protein kinase Kit & NM\_000222.2 \\
    KRAS & KRas proto-oncogene GTPase & NM\_033360.2 \\
    MAPK1 & Mitogen-activated protein kinase 1 & NM\_002745.4 \\
    MAP2K1 & Mitogen-activated protein kinase kinase 1 & NM\_002755.3 \\
    MTHFR & Methylenetetrahydrofolate reductase & NM\_005957.4 \\
    MTOR & Serine/threonine-protein kinase mTOR & NM\_004958.3 \\
    NRAS & Neuroblastoma RAS viral oncogene homolog & NM\_002524.3 \\
    PDGFRA & Platelet-derived growth factor receptor alpha & NM\_006206.4 \\
    PIK3CA & Phosphatidylinositol-4,5-bisphosphate 3-kinase catalytic subunit alpha & NM\_006218.2 \\
    PTEN & Phosphatase and tensin homolog & NM\_000314.4 \\
    STAT1 & Signal transducer and activator of transcription 1 & NM\_007315.3 \\
    STAT3 & Signal transducer and activator of transcription 3 & NM\_139276.2 \\
    TP53 & Tumor protein P53 & NM\_000546.5 \\
    TYMP & Thymidine phosphorylase & NM\_001113755.2 \\
    TYMS & Thymidylate synthetase & NM\_001071.2 \\
    UGT1A1 & Uridine diphosphate (UDP)-glucuronosyl transferase 1A1 & NM\_000463.2\\
    \hline
    \end{tabular}
\end{table}

%%%%%%%%%%%%%%%%%%%%%%%%%%%%%%%%%%%%%%%%%%%%%%%%%%%%%%%%%%%%%%%%%%%%%%
\section{Variant Calling Pipeline}
\label{sec:VariantCallingPipeline}

Reads that passed the Illumina Chastity filter were aligned to the GRCh37/hg19 human reference genome using the BWA mem algorithm (version 0.5.9) with default parameters, and the alignments were processed and converted to the BAM format using SAMtools (version 0.1.18). To assess the true positive rate of germline variant calling in FFPE tumours, variant calling was performed separately for blood and tumour sequencing libraries. The SAMtools \texttt{mpileup} function \texttt{(samtools mpileup -BA -d 500000 -L 500000 -q 1)} was used to generate pileup files for all target bases followed by the VarScan2 \texttt{mpileup2cns} (version 2.3.6) function with parameter thresholds of variant allele frequency $\geq$ 10\% and Phred-scaled base quality score $\geq$ 20 \mbox{\texttt{(--min-var-freq 0.1 --p-value 0.01 --strand-filter 0 --output-vcf}} \\ \texttt{--variants --min-avg-qual 20)}. As for comparison of variant calls between FFPE tumour replicates, variant calling was performed on tumour-normal pairs using similar parameters for the SAMtools \texttt{mpileup} function, but followed by the VarScan2 \texttt{somatic} (version 2.3.6) function. Variant calls were filtered using the VarScan2 \texttt{fpfilter} function with fraction of variant reads from each strand $\geq$ 0.1 and default thresholds for other parameters. SnpEff (version 4.2) was used for variant annotation and effect prediction whereas the SnpSift package in SnpEff was used to annotate variants with databases such as dbSNP (b138), COSMIC (version 70), 1000 Genomes Project, ClinVar, and ExAC (release 0.3) for interpretation.

%%%%%%%%%%%%%%%%%%%%%%%%%%%%%%%%%%%%%%%%%%%%%%%%%%%%%%%%%%%%%%%%%%%%%%
\section{Data Analysis}
\label{sec:DataAnalysis}

Coverage depth was measured using bedtools (version 2.25.0) and per-base metrics were obtained using bam-readcount (https://github.com/genome/). Statistical analyses and data visualization were performed using R (version 3.3.2) and associated open-source packages. Manual review of PGx variants were carried out using the Intergrative Genomics Viewer (IGV, version 2.3). \textit{Note: be more specific on how the data is generated}


%%%%%%%%%%%%%%%%%%%%%%%%%%%%%%%%%%%%%%%%%%%%%%%%%%%%%%%%%%%%%%%%%%%%%%

\endinput
